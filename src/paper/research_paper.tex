\documentclass[11pt, a4paper, leqno]{article}
\usepackage{a4wide}
\usepackage[T1]{fontenc}
\usepackage[utf8]{inputenc}
\usepackage{float, afterpage, rotating, graphicx}
\usepackage{epstopdf}
\usepackage{longtable, booktabs, tabularx}
\usepackage{fancyvrb, moreverb, relsize}
\usepackage{eurosym, calc}
% \usepackage{chngcntr}
\usepackage{amsmath, amssymb, amsfonts, amsthm, bm}
\usepackage{caption}
\usepackage{mdwlist}
\usepackage{xfrac}
\usepackage{setspace}
\usepackage{xcolor}
%\usepackage{subcaption}
\usepackage{minibox}
% \usepackage{pdf14} % Enable for Manuscriptcentral -- can't handle pdf 1.5
% \usepackage{endfloat} % Enable to move tables / figures to the end. Useful for some submissions.


\usepackage{threeparttable}
\usepackage{subfigure}
\usepackage{lscape}
\usepackage{datetime}


\usepackage{natbib}
%\bibliographystyle{rusnat}
\bibliographystyle{elsarticle-harv}




\usepackage[unicode=true]{hyperref}
\hypersetup{
    colorlinks=true,
    linkcolor=black,
    anchorcolor=black,
    citecolor=black,
    filecolor=black,
    menucolor=black,
    runcolor=black,
    urlcolor=black
}

\widowpenalty=10000
\clubpenalty=10000

\setlength{\parskip}{1ex}
\setlength{\parindent}{0ex}
\setstretch{1.5}

\newtheorem{hypothesis}{HYPOTHESIS}
\linespread{1.5}


\begin{document}

\title{Drivers of FinTech Lending\thanks{Wenxin Hu, University Bonn. Email: \href{mailto:s6wehuuu@uni-bonn.de}{\nolinkurl{s6wehuuu [at] uni-bonn [dot] de}}.}}

\author{Wenxin Hu}

\date{
\today
}

\maketitle
%\thispagestyle{empty}

\begin{abstract}
\noindent
The use of big data and machine learning are labeled as “FinTech” to account for the use of new technologies in finance. And peer-to-peer(P2P) lending is a practice of this new business model. We use data from one leading lending platforms on US consumer credit market, Lending Club, to explore the main drivers of FinTech lending. Our main hypotheses focus on four aspects: 1) financial crisis; 2) market structure; 3) technology; 4) demographics. We find that FinTech lending has been spurred by the financial crisis. High market concentration appears to deter entry and expansion of the FinTech. Also we find that past technology may require some time to affect future FinTech lending positively. Finally, we find a positive impact of demographic variables, such as unemployment rate and proportion of young population.t here.
\end{abstract}
\vspace{0.5cm}
\textit{Keywords:} FinTech lending, financial crisis, market structure, technology, Internet, socio-economic demographics


%\thispagestyle{empty} %\setcounter{page}{0}
\newpage
%\thispagestyle{empty}
\tableofcontents
\newpage
%\thispagestyle{empty}
\listoffigures
\newpage
%\thispagestyle{empty}
\listoftables
\newpage


\pagenumbering{arabic}
\section{Introduction}

%\paragraph{Motivation.}

Fintech lending are online intermediaries that match lenders with borrowers. Here we use the P2P lending because we study the drivers of this new business model.

In this paper we explore the main drivers behind the rapid expansion of demand for credit from fintech lending via P2P online platforms. These platforms allow matching savers with borrowers who need consumer and business loans. The volume of fintech lending has been growing rapidly.  

The entry of the fintech lending platforms has coincided with the global financial crisis and its rapid expansion has happened as the banking sector was undergoing important structural transformations. In reaction to the crisis, banks were deleveraging, consolidating, reducing their credit supply and cutting costs by closing branches. The stock of consumer loans orginated by banks has fallen for a number of years, before starting to pick up at 2011(see figure 2). The objective of this paper is to explore the idea that fintech lending platforms target underserved borrowers and succeed to enter markets from which banks have withdrawn in the wake of the crisis.  

\begin{center}
- Figure \ref{fig:LC} about here -
\end{center}

We outline four main hypotheses to explain the driven factors for fintech lending. 
\textit{First hypothesis} is that fintech credit demand has been stimulated by the global financial crisis, as numerous banks have failed and others have been forced to deleverage and reduce their credit supply. \textit{Second hypothesis} is about high market concentration und barriers to entry.  \textit{Third hypothesis} links the speed of the development of the fintech lending to Internet use of borrowers that have access to P2P lending platforms. \textit{Fourth hypothesis} is related to the characteristics of demographics and socio-economics, such as education, personal income, unemployment rate, youth ratio, and poverty.

To test the above hypothesis is difficult because the fintech lending has coincided with the post-crisis period, increased concentration of bank sector and closing of bank branches. Our identification strategy relies on the exploration of the geographic heterogeneity of fintech lending at the county level. To undertake this empirical analysis, we aggregate data from Lending Club and measure their expansion by aggregating the volume and number of loans provided by this platform. 
Several papers have started to explore the determinants of the entry and expansion of Fintech. Rau (2017) explores the determinants of the development of crowdfunding at the global level and finds that quality of regulation, financial development, and ease of Internet access are all positively related to crowdfunding volume while the ease of doing business is negatively associated. While our paper focuses on fintech that are P2P lending platforms. Butler et al. (2014) explore how borrowers choose between traditional and alternative sources of finance and show that borrowers who reside in areas with good access to bank finance request loans with lower interest rates on P2P lending platforms.

\section{Data}
To construct variables about diffusion of fintech lending, we rely on loan book data from Lending Club. Unfortunately, the data from Lending Club doesn't have an clear indicator for United States counties. Therefore, we use the first 3 digits of provided zip code to construct a indicator called county. However, this indicator does not exactly represent United State counties, because zip codes are not drawn according to state or county boundaries. In order to have a wider variation among observations, we use this indicator to roughly assign county level geographics area. The "county" we use in the following contents means this indicator to simplify variable explanation. We have 10307 observation points, corresponding to a total volume of funded loans equal to 26.081 billion, starting from January 2007 to December 2017. 

Since loan book data provide information on each borrower's county, we aggregate this data at the year-county level to construct two measure of fintech lending diffusion: volume and number of fintech lending.
Table \ref{sumcol} shows the total volume of funded loans, the total number of loans provided by Lending Club from 2007 to 2017.
\begin{center}
- Table \ref{sumcol} about here -
\end{center}

For our regressions, several data of the demographic variables are only provided by state level. So we aggregate annual data of fintech lending for each county and sort them by state to where it belongs, then merge our dataset with other datasets that contain our explanatory variables. For the variables that are provided at state level, we can only use it for multiple counties in one state.  Our specification accounts for a large number of characteristics that could influence the expansion of the fintech lending. 

Table \ref{vardef} provides exact definitions of all variables, Table \ref{sumstat} provides summary statistics and Table \ref{table:corr} provides the correlation matrix.
 
\textbf{Measuring financial crisis.}

The FDIC Summary of Deposits allows us to calculate the number of bank failures in each county. To do this, we merge FDIC failed bank list with the data on branches of these banks in each county from the FDIC Summary of Deposits to compute the number of bank failures during the analyzed period. 
\begin{center}
- Figure \ref{fig:failure} about here -
\end{center}

\textbf{Measuring market structure}

The FDIC Summary of Deposits also allows us to calculate a number of market structure variables that are used as proxies for entry barriers. In particular, we compute Herfindahl-Hirschman Index(HHI) at the county level to measure the market concentration.

\begin{center}
- Figure \ref{fig:HHI} about here -
\end{center}

\textbf{Measuring technology development.} 

We use data "State Broadband Initiative" from the National Telecommunications and Information Administration (NTIA) to calculate the Internet penetration at the state level. 
This data allows us to compute the percent of state population with access to optical fiber technology. And these data are available from a survey, we need to interpolate data for year 2008, 2014 and 2016 since there are no such surveys conducted in the three years.

\textbf{Measuring socio-economic demographics.} 

To measure socio-economic demographics, we rely on several data resources for our related explanatory variables: we use the GDP data from the U.S. Bureau of Economic Analysis to calculate personal income per capita; we use data of unemployment rate from the Bureau of labor statistics; to compute educational attainment, youth ratio and poverty ratio from the US Census Bureau. By educational attainment, we use the proportion of population with at least a bachelor degree. By youth ratio, we use the percentage of population whose age is between 18 and 24.

\textbf{Comparing fintech lending to traditional banking.}

In order to find whether these explanatory variables drive only fintech lending or drive fintech lending and traditional banking simultaneously, we use "State­ level Household Debt Statistics" from Federal Reserve Bank of New York to make this comparison. 
Figure \ref{fig:TB} illustrates the Traditional Banking Loan growth (in billions of dollars) over the period 2007 to 2017. We use it and lending club loan amount to calculate the difference between them. Then we try to explore some specific variables for FinTech lending only by this "difference" variable.

\begin{center}
- Figure \ref{fig:TB} about here -
\end{center}
%We set a new variable named "Fintech ratio", which is the ratio of loan amount provided by fintech lending to traditional bank loans, by which we try to explore some specific variables for fintech lending only.

\section{Analysis}
The analysis starts with exploring potential drivers of the expansion of fintech lending. Then we try to find some specific drivers for fintech lending comparing to traditional banking.

For each variable we run regressions with and without state fixed effects. The specification without state fixed effects has advantage that level differences across states are not absorbed by the fixed effects. The specification with state fixed effects has the advantage that it captures all time-invariant unobserved state characteristics and mitigates endogeneity concerns.

\subsection{Hypothesis}
\begin{hypothesis}
\textbf{Fintech lending expanded faster in markets that were more affected by the Global Financial Crisis.}
\end{hypothesis}

The entry and the early expansion of fintech lending might have been spurred by the incidence of financial crisis of 2007-08. Total consumer credit significantly decreased in the years 2008-2010 and credit rationing may have spurred the demand for alternative forms of financing. Survey evidence suggests that the impact of the Global Financial Crisis could be long-term and even permanent due to borrowers' mistrust of the traditional banking sector. In response to the question about the main advantages of borrowing from a fintech lending platform, 54\% of Funding Circle's borrowers responded that it is "Not my bank". The only response that was more popular was ‘speed of securing finance’ (58\%). This suggests that the choice to use P2P lending platforms could be permanent even when banks deleverage and pick-up their credit supply.

\begin{hypothesis}
\textbf{Fintech lending expanded faster in areas with low market concentration, which is related to barriers to entry.}
\end{hypothesis}
The entry and early expansion of fintech lending platforms could also be related to the nature of the banking competition and barriers to entry. The banking sector is characterized by monopolistic competition due to high barriers to entry, switching costs and strong brand loyalty (Claessens and Laeven, 2004; Shy, 2002; Kim et al., 2003). Philippon (2015) demonstrates that the cost of financial intermediation in the US has remained unchanged since the 19 centuries. This fact is astonishing in the context of rapid progress in the communication and formation technologies that should have driven down the price of financial services for end users. Hence, the entry of new fintech suppliers could be needed to improve the provision of financial services and disrupt traditional banks. Indeed, online lenders argue that their operating expenses are much lower than those banks due to the extensive use of new technologies as well as absence of legacy problems and costly branch networks.

There are two conflicting views on the relationship between market structure and competition. On the one hand, market concentration assumes that firms' price choices and thus their profitability. If customers from these markets decide to switch to online platforms, we could expect a positive relationship between market concentration and demand for fintech lending. On the other hand, the efficiency hypothesis stipulates that a firm that operates more efficiently than its competitors should gain higher profits and hence, concentrated markets could be a result of the survival of the most efficient banks. Empirical papers find mixed evidence about the relationship between market concentration and competition. Bolt and Humphrey (2015) analyse the US banking market and find no correlation between market structure and competition (measured by Lerner Index and Panzer and Ross H- statistic). Claessens and Laeven (2004) analyze a large sample of countries and find that concentrated markets are characterized by more competition.

\begin{hypothesis}
\textbf{Fintech expanded faster where people have easier accesses to Internet.}
\end{hypothesis}
It is important to note that fintech lending is done via a web-page, the entry into new market is not physical and simply reflects a borrower's approach to asking credit from a P2P lending platform.  Several papers have started to explore the technological determinants of the entry and expansion of FinTechs. Haddad and Hornuf (2016) find that FinTech startups are created in countries where the latest technology is readily available and people have more mobile telephone subscriptions. DeYoung et al. (2007) and Hernando et al. (2007) analyze the impact of the adoption of online banking on banks’ profitability and find that the Internet channel is a complement to rather than a substitute for physical branches. 

\begin{hypothesis}
\textbf{The expansion of the fintech lending is faster in areas with more educated, young population, higher unemployment rate and lower personal income.}
\end{hypothesis}
Besides the Internet access, whether borrowers decide to apply credit from fintech lending is also an important point that we would like to explore. Hence we focus on these demographic factors that could influence the decision, such as educational attainment, personal income, unemployment rate, youth ratio and poverty.

If bank customers wanted to switch to an online lending platform, they would need to build trust in it and incur learning costs about P2P lending and as well as transaction costs to set up their profile and describe their loan. Learning and psychological switching costs could depend on local county level characteristics. Indeed, survey evidence shows that younger and more educated individuals were more likely to adopt electronic banking in the 90s (Kennickell and Kwast, 1997). The same survey shows that the most popular source of information for saving and borrowing decisions is calling around friends, relatives, and colleagues. Hence, spatial effects could reflect human interactions that lower learning costs and speed up technological diffusion. We expect that counties with higher unemployment rate, and higher proportion of educated, young people, should have higher levels of fintech lending.

\subsection{Drivers of fintech lending}

\paragraph{Regression model} 
To test our main hypotheses on the expansion of fintech lending, we specify the following model:
\begin{equation}\begin{split}
log(loan\_amount)_{i,j,t} = & \beta_0 + \tau_{y} + \alpha_{j} + \beta_1 financial\ crisis_{i,j,t} \\ 
+ &  \beta_2 market\ structure_{i,j,t} + \beta_3 technology_{j,t}\\ 
+ &  \beta_4 demographics_{j,t} +\epsilon_{i,j,t},
     \end{split}\end{equation}

where $log(loan\ amount)_{i,j,t}$ represents the natural logarithms of loan amount provided by Lending Club in county $i$ of state $j$ in year $t$; $\beta_1$ represents the coefficient of our independent variables regarding the bank failure; $\beta_2$ represent the coefficient of independent variables regarding the market structure; $\beta_3$ is the coefficient of our variables regarding technology development; $\beta_4$ is the coefficient of our independent variables for social-economic demographic variables that could capture demand for fintech lending (see table \ref{vardef} for the detailed list of observed independent variables). $\tau_{y}$, $\alpha_{j}$ for year and state fixed effects respectively. and $\epsilon_{i,j,t}$ the error term. 

We also use the number of loans provided by Lending Club as dependent variables and compare the two results.

\paragraph{Results}
Regression results are shown in Table \ref{table:baseline}, we present our empirical findings for the adoption of fintech lending (in terms of natural logarithms of volume and number of loans respectively) and discuss the impact of a change in the explanatory variables. 

Our findings show that in all specifications, the bank failure is statistically significant and has a positive effect on fintech lending expansion. The significantly positive coefficient of 0.365 implies that one bank failure is on average associated with a 36.5 percent change in fintech lending amount. Its impact would also be consistent with the idea that fintech lending platforms enter markets that are more affected by the Global Financial Crisis.

Then we turn to the market structure. We find that high market concentration (HHI) of traditional banks appears to deter the adoption of the fintech lending. This significantly negative coefficient of -0.9325 implies that 1 percentage point increase in the Herfindahl-Hirschman Index will decrease the fintech lending amount by 0.93 percent. As explained earlier, the presence of branches could be considered as an advertisement strategy that increases brand loyalty to banks. This aligns our barriers to entry hypothesis. These results support our financial crisis hypothesis that fintech lending platforms are used by customers that are underserved by traditional banks. 

We now turn our attention to variables that capture the geographic heterogeneity in terms of technology, measured by the penetration of Internet by each state. This variable is insignificant with and without fixed effect, which is inconsistent with our hypothesis. This is an interesting result because it means that the entry of the fintech lending is not a technological phenomenon, but rather an economic one. 

As expected, among economic demographic variables, we find that the expansion of fintech lending is faster in states with higher unemployment rate and share of young population. The significantly positive coefficient of unemployment implies that people who are unemployed have a greater demand for fintech lending. And the coefficient of young population is significantly positive when state fixed effects are included, reflecting a higher willingness to adopt new technologies despite of differences between states. Educational attainment, personal income and poverty ratio are all insignificant with or without state fixed effect.

The results using the number of fintech lending as dependent variable are the same as above. 

\subsection{Fintech lending vs traditional banking}
\paragraph{Regression model} 
In order to explore the specific drivers of fintech lending, we use the difference between natural logarithms of fintech loan amount and natural logarithms of traditional bank loan as explained variable, other explanatory variables stay the same. We specify the following model:
\begin{equation}\begin{split}
log(loan\_amnt)_{i,j,t} - log(bank\_loan)_{j,t} = & \beta_0 + \tau_{y} + \alpha_{j} + \beta_1 financial\ crisis_{i,j,t} \\ 
+ &  \beta_2 market\ structure_{i,j,t} + \beta_3 technology_{j,t}\\ 
+ &  \beta_4 demographics_{j,t} +\epsilon_{i,j,t},
     \end{split}\end{equation}


Additionally, since the most common use of fintech lending is to mitigate short term liquidity shock, So it is reasonable to compare fintech lending amount to a more similar business within traditional banks. Thus, we use credit card loan provided by traditional bank to be divided by fintech lending amount, and do the regression again.  

\paragraph{Results}
Regression results are shown in Table \ref{table:compare}, we change the explained variable to the difference between natural logarithms of fintech lending amount and natural logarithms of traditional bank loan, run the regressions always with year fixed effect and state fixed effect, and try to find specific drivers for fintech lending. 

We find that variables with statistically significant coefficients of total traditional banking loan are the same as variables of credit card loan of traditional banking. 

Our findings show that in all specifications, the bank failure is statistically significant and has a positive effect on fintech lending expansion. And market concentration (HHI) still have a negative impact on fintech. And the coefficients of unemployment, young population are significantly positive, the same as in the last regression.


\subsection{Robustness analysis}

So far, we have known the possible drivers of FinTech lending. In this context, we estimate our preferred specification by two different robustness tests. The analysis results are in Table \ref{table:robustness}. 

\paragraph{The first estimation is to argue that our model is not suffer from the reverse causality.} 
We use all one-year lagged explanatory variables to estimate the logarithm of amount and number of FinTech lending. According to our identification strategy, financial crisis, change of market structure, development of Internet technology and some socio-economic demographics from the past may affect the current development of FinTech lending. However, they should not affect FinTech lending if they are from the future. Therefore, in this lag model, we try to avoid the reverse effect from dependent variable by including lags. If the coefficients of interest are still significant, we can say the model is unlikely to have a serious reverse causality problem which may lead to internal validity problem of the causal effect.

\paragraph{Regression model} 
In order to explore the existence of reverse causality, we specify the following model:
\begin{equation}\begin{split}
log(loan\_amount)_{i,j,t} = & \beta_0 + \tau_{y} + \alpha_{j} + \beta_1 financial\ crisis_{i,j,t-1} \\ 
+ &  \beta_2 market\ structure_{i,j,t-1} + \beta_3 technology_{j,t-1}\\ 
+ &  \beta_4 demographics_{j,t-1} +\epsilon_{i,j,t-1},
     \end{split}\end{equation}

Time and state fixed effects are included. All variables are defined as in the previous regressions.

\paragraph{Results}
As shown by the coefficients of all variables in the column (1) and (2) of Table \ref{table:robustness} with FinTech lending volume and number, respectively, we can see that significance levels have no change from our baseline model. Bank failure, unemployment and young have a significantly positive relationship with FinTech lending development and HHI has a significantly negative relationship with FinTech lending development. The result is in line with our previous results and it would support the robustness of our findings.

One thing is also interesting. With one-year lag, we find a significantly positive relation between Internet and FinTech lending development. This result actually is in line with our original hypothesis. One possible interpretation is Internet and technology would need some time to be accepted by the majority of households.

\paragraph{In the second estimation, we add the interaction term of bank failures and HHI in the original model in order to be more specific about the effect channels of financial crisis and market structure.} 
From previous study, we know there might be an endogeneity problem related to the fact that market structure and branch density have been profoundly affected by the financial crisis of 2007/2008. Counties that have witnessed increased market concentration are also those that have suffered the most from the financial crisis. If the coefficient of interaction term is significant, we can prove the conditional marginal effects of financial crisis and market structure exist.

\paragraph{Regression model} 
In order to explore the effect channels of financial crisis and market structure, we specify the following model:
\begin{equation}\begin{split}
log(loan\_amount)_{i,j,t}= & \beta_0 + \tau_{y} + \alpha_{j} + \beta_1 financial\ crisis_{i,j,t} \\ 
+ & \beta_2 market\ structure_{i,j,t} + \beta_3 technology_{j,t} \\ 
+ & \beta_4 demographics_{i,j,t}  \\
+ & \beta_5 financial\ crisis_{j,t} \times  market\ structure_{i,j,t} + \epsilon_{i,j,t},
     \end{split}\end{equation}

where time and state fixed effects are included and $\beta_5$ is the interaction term of crisis and market structure. All variables are defined as in the previous regressions.

\paragraph{Results}
As shown by the coefficients of all variables in the column (3) and (4) of Table \ref{table:robustness} with FinTech lending volume and number, respectively, the significantly positive coefficient of interaction term $financial\ crisis_{i,j,t} \times  market\ structure_{i,j,t}$ shows that higher market concentration level is related to a stronger effect of financial crisis on FinTech lending expansion within state. In particular, if the HHI increases by 10 percentage points, the elasticity between financial crisis and FinTech lending will be increased by 7 percentage points. This is economically highly relevant. 

Figure \ref{fig:Margin} illustrates the marginal effects of this regression, corresponding to column (3) of Table \ref{table:robustness}. As shown in panel (a), the relation between FinTech lending development and financial crisis is stronger for counties with a relatively high market concentration level. Panel (b) shows that such a market concentration level is associated with a lower FinTech development only if the financial crisis in the county is not relatively serious (about the mean value across the sample, which is zero on the x-axis, or lower). In other words, high market concentration level in one county is not enough to be a barrier to entry for FinTech lendings if the financial crisis is considered relatively serious. The result is in line with our previous results and hypotheses and it would support the robustness of our findings.

\begin{center}
- Figure \ref{fig:Margin} about here -
\end{center}

\begin{center}
- Table \ref{table:robustness} about here -
\end{center}


\section{Summary and Conclusion}

This paper is a first attempt to explore the drivers of FinTech lending in the US. We have proposed four hypotheses related to 1) financial crisis; 2) market concentration; 3) technology development; 4)demographic characteristics. Our findings are broadly consistent with the idea that FinTech lending platforms have entered markets that are underserved by banks. We also find that the entry of FinTech lending is constrained by the high market concentration, which has been interpreted as entry barriers. States with higher share of young population and higher unemployment rate experience higher growth of fintech lending.

What we find interesting is that we do not find any significant impact of Internet penetration on the expansion of fintech lending in the baseline regression although it requires access to Internet. However, we discover its lagged variable could have positive effect on FinTech lending. Furthermore, the robustness test including interaction term improves the explanation of variable hypotheses and effects. Although our model still need improvement, it seems quite plausible that causal effects from financial crisis, market structure, young population and unemployment rate exist.

\nocite{*}

%%%%%%%%%%%%%%%%%%%%%%%%%%%%%%%%%%%%%%%%%%%%%%%%%%%%%%%%%%%%%%
\newpage
\linespread{1.25}\selectfont
\setlength{\bibsep}{12pt}
\bibliography{refs}

\clearpage
\pagenumbering{Roman}
\section*{Appendix: Figures}

\begin{figure}[!h!]
\centering
    \subfigure[Volumne]
     {\includegraphics[scale=0.5]{../../out/figures/figure2_lc_volume.eps}}
    \subfigure[Number]
     {\includegraphics[scale=0.5]{../../out/figures/figure2_lc_number.eps}}
\caption{Lending Club loan volume and number development}\label{fig:LC}
\vspace{0.25cm}
\hspace{0.5cm} \parbox{15cm}{\scriptsize The first panel illustrates the Lending Club loan volume growth (in billions of dollars) over the period 2007 to 2017. The second panel shows the respective quantity growth. Source: Lending Club.}
\end{figure}

\begin{figure}[!h!]
\centering {\includegraphics[scale=0.75]{../../out/figures/figure5_failure.eps}}
\caption{Bank Failures development}\label{fig:failure}
\vspace{0.25cm}
\hspace{0.5cm} \parbox{15cm}{\scriptsize The panel illustrates the county averaged bank failures over the period 2007 to 2017. Source: Summary of Deposits}
\end{figure}

\begin{figure}[!h!]
\centering {\includegraphics[scale=0.75]{../../out/figures/figure4_hhi.eps}}
\caption{HHI development}\label{fig:HHI}
\vspace{0.25cm}
\hspace{0.5cm} \parbox{15cm}{\scriptsize The panel illustrates the county averaged HHI over the period 2007 to 2017. Source: Summary of Deposits}
\end{figure}

\begin{figure}[!h!]
\centering {\includegraphics[scale=0.75]{../../out/figures/figure3_tradition.eps}}
\caption{Traditional banking loan amount development}\label{fig:TB}
\vspace{0.25cm}
\hspace{0.5cm} \parbox{15cm}{\scriptsize The panel illustrates the Traditional Banking Loan growth (in billions of dollars) over the period 2007 to 2017. Source: FRBNY Consumer Credit Panel / Equifax}
\end{figure}
\begin{figure}[!h!]
\centering
    \subfigure[Cond. marginal effect of bank failures on log(loan amount)]
     {\includegraphics[scale=0.75]{../../out/figures/figure1_margin_failure.eps}}
    \subfigure[Cond. marginal effect of HHI on log(loan amount)]
     {\includegraphics[scale=0.75]{../../out/figures/figure1_margin_hhi.eps}}
\caption{Conditional marginal effects of Bank Failures and HHI}\label{fig:Margin}
\vspace{0.25cm}
\hspace{0.5cm} \parbox{15cm}{\scriptsize The estimated regression model is Equation 4. The dashed lines represent the mean minus standard
deviation, mean, and mean plus standard deviation of the x-variable.}
\end{figure}



\clearpage
\section*{Appendix: Tables}



% DATASET
%\begin{landscape}
\begin{center}
\begin{table}[!h!] \caption{Lending Club dataset(loan volumes, number of loans and counties) \label{sumcol}}
\begin{threeparttable}
\begin{tabularx}{\linewidth}{X*{7}{l c c c c c c}}
\hline
\multicolumn{7}{l}{} \\
year & 2007 & 2008 & 2009 & 2010 & 2011 & 2012 \\
\hline
Fintech Lending Volume & 0.005 & 0.021 & 0.052 & 0.134 & 0.262 & 0.718 \\
Fintech Lending Number & 605 & 2422 & 5383 & 12692 & 21722 & 53360 \\
County Number & 239 & 525 & 637 & 719 & 761 & 787 \\
\hline
\multicolumn{7}{l}{} \\
\multicolumn{7}{l}{} \\
\hline
\multicolumn{7}{l}{} \\
year & 2013 & 2014 & 2015 & 2016 & 2017 & \\
\hline
Fintech Lending Volume & 1.982 & 3.502 & 6.415 & 6.398 & 6.583 & \\
Fintech Lending Number & 134773 & 235527 & 420947 & 434259 & 443434 & \\
County Number & 820 & 837 & 872 & 868 & 864 & \\
\hline
\end{tabularx}
\begin{tablenotes}
      \scriptsize
      \item Notes: Fintech Lending Volume is in billions of dollars. We classify counties by first 3 digits of provided zipcode.
      \end{tablenotes}
\end{threeparttable}
\end{table}
\end{center}
%\end{landscape}



%%% VARIATBLE DEFINITIONS
\begin{table}[!h!]\centering \caption{Variable definitions and data sources \label{vardef}}
\begin{tabularx}{\linewidth}{|p{5cm}|p{8cm}|p{1.7cm}|}
\hline
\textbf{Variable} & \textbf{Definition and data source} & \textbf{Expected sign} \\ 
\hline
\multicolumn{3}{|l|}{\textbf{Dependant variables}} \\ 
\hline
Fintech Lending (volume) & Logarithm of loan volume from Lending Club at county level(dollars). &  \\
log(loan\_amnt) & Source: Lending Club &  \\ 
\hline
Fintech Lending (number)  & Logarithm of the number of loans from Lending Club at county level. &   \\
log(loan\_no) & Source: Lending Club &  \\
\hline
Traditional Banking Loan (Total) & Logarithm of total household debt balance in traditional banking (dollars).  &   \\ 
log(bank\_loan) & Source: FRBNY Consumer Credit Panel / Equifax &   \\ 
\hline
Traditional Banking Loan (Credit Card) & Logarithm of credit card debt balance in traditional banking (dollars).  &   \\ 
log(bank\_loan) & Source: FRBNY Consumer Credit Panel / Equifax &   \\ 
\hline
\multicolumn{3}{|l|}{\textbf{Financial crisis}} \\ \hline
failure & Number of bank failures at county level. Source: Summary of Deposits.&  + \\ \hline
\multicolumn{3}{|l|}{\textbf{Market Structure}} \\ \hline
hhi & HHI for market concentration at county level. Source: Summary of Deposits.&  - \\ \hline
\multicolumn{3}{|l|}{\textbf{Technology}} \\ \hline
internet & Uses the Internet. Source: NTIA. &  + \\ \hline
\multicolumn{3}{|l|}{\textbf{Socio-economic Demographics}} \\ \hline
income & Per capita personal income(thousand of dollars). Source: DepaU.S. Bureau of Economic Analysis. &  ? \\ \hline
unemployment & Unemployment rate(\%). Source: Bureau of Labor Statistics.&  + \\ \hline
bachelor & Propotion of the population with bachelor degree(\%). Source: US Census Bureau.&  + \\ \hline
young & Propotion of young population(age between 18 and 24)(\%). Source: US Census Bureau. &  + \\ \hline
poverty & Proportion of population below poverty level. Source: US Census Bureau. & ? \\ \hline
\end{tabularx}
\end{table}




%%% SUMMARY
\begin{center}
\begin{table}[!h!] \caption{Summary statistics \label{sumstat}}
\begin{threeparttable}
\begin{tabularx}{\linewidth}{X*{6}{l c c c c c}}

\input{../../out/tables/table1_sumstat.tex}

\end{tabularx}
  \begin{tablenotes}
      \scriptsize
      \item Notes: This table shows summary statistics for all variables used in the analysis. The sample includes 833 counties over the period 2007 to 2017. See Table 2 for a detailed description of all variables.
      \end{tablenotes}
\end{threeparttable}
\end{table}
\end{center}



%%% CORRELATION
\begin{landscape}
%\begin{center}
\begin{table}[!h!] \caption{Correlation Matrix \label{table:corr}}

\input{../../out/tables/table2_corr.tex}

\end{table}
%\end{center}
\end{landscape}



%%% REGRESSION 1
%\begin{landscape}
\begin{center}
\begin{table}[!h!] \caption{Panel model for the FinTech lending development \label{table:baseline}}
\begin{threeparttable}
\begin{footnotesize}
\begin{tabularx}{\textwidth}{Xrrrr}

\begin{equation}\begin{split}
log(loan\_amount)_{i,j,t} = & \beta_0 + \tau_{y} + \alpha_{j} + \beta_1 financial\ crisis_{i,j,t} \\ 
+ &  \beta_2 market\ structure_{i,j,t} + \beta_3 technology_{j,t}\\ 
+ &  \beta_4 demographics_{j,t} +\epsilon_{i,j,t},
     \end{split}\end{equation}


\end{tabularx}
\end{footnotesize}
  \begin{tablenotes}
      \scriptsize
      \item Notes: Dependent variable are the logarithm of amount and number of FinTech lending. Variable definitions are provided in Table 2. Standard errors are clustered on county level. We show p-values in parentheses. The ***, ** and  * stand for significant coefficients at the 1\%, 5\%, and  10\% levels, respectively.
      \end{tablenotes}
\end{threeparttable}
\end{table}
\end{center}
%\end{landscape}



%%% REGRESSION 2
%\begin{landscape}
\begin{center}
\begin{table}[!h!] \caption{Comparison between FinTech and Traditional Lending \label{table:compare}}
\begin{threeparttable}
\begin{footnotesize}
\begin{tabularx}{\textwidth}{Xccccc}

\begin{equation}\begin{split}
log(loan\_amnt)_{i,j,t} - log(bank\_loan)_{j,t} = & \beta_0 + \tau_{y} + \alpha_{j} + \beta_1 financial\ crisis_{i,j,t} \\ 
+ &  \beta_2 market\ structure_{i,j,t} + \beta_3 technology_{j,t}\\ 
+ &  \beta_4 demographics_{j,t} +\epsilon_{i,j,t},
     \end{split}\end{equation}


\end{tabularx}
\end{footnotesize}
  \begin{tablenotes}
      \scriptsize
      \item Notes: Dependent variable are (1)logarithm of amount of FinTech lending; (2)logarithm of amount of total traditional banking lending; (3)difference between (1) and (2); (4)logarithm of amount of traditional banking lending in credit card only; (5)difference between (1) and (4). Variable definitions are provided in Table 2. Standard errors are clustered on county level. We show p-values in parentheses. The ***, ** and  * stand for significant coefficients at the 1\%, 5\%, and  10\% levels, respectively.
      \end{tablenotes}
\end{threeparttable}
\end{table}
\end{center}
%\end{landscape}



%%% REGRESSION 3
%\begin{landscape}
\begin{center}
\begin{table}[!h!] \caption{Robustness Test \label{table:robustness}}
\begin{threeparttable}
\begin{footnotesize}
\begin{tabularx}{\textwidth}{Xrrrr}

\input{../../out/analysis/reg3_robustness.tex}

\end{tabularx}
\end{footnotesize}
  \begin{tablenotes}
      \scriptsize
      \item Notes: Dependent variable are the logarithm of amount and number of FinTech lending. Variable definitions are provided in Table 2. "L." means that it is a 1-year lag variable. Standard errors are clustered on county level. We show p-values in parentheses. The ***, ** and  * stand for significant coefficients at the 1\%, 5\%, and  10\% levels, respectively.
      \end{tablenotes}
\end{threeparttable}
\end{table}
\end{center}
%\end{landscape}



\end{document}
