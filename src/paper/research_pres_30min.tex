\documentclass[11pt]{beamer}
% \documentclass[11pt,handout]{beamer}
\usepackage[T1]{fontenc}
\usepackage[utf8]{inputenc}
\usepackage{float, afterpage, rotating, graphicx}
\usepackage{epstopdf}
\usepackage{longtable, booktabs, tabularx}
\usepackage{fancyvrb, moreverb, relsize}
\usepackage{eurosym, calc}
\usepackage{amsmath, amssymb, amsfonts, amsthm, bm} 
\usepackage{caption}
\usepackage{subfigure}



\usepackage{natbib}
\bibliographystyle{rusnat}


\hypersetup{colorlinks=true, linkcolor=black, anchorcolor=black, citecolor=black, filecolor=black, menucolor=black, runcolor=black, urlcolor=black}

\setbeamertemplate{footline}[frame number]
\setbeamertemplate{navigation symbols}{}
\setbeamertemplate{frametitle}{\centering\vspace{1ex}\insertframetitle\par}
\usefonttheme[onlylarge]{structuresmallcapsserif}


\begin{document}

\title{Drivers of FinTech Lending}

\author[Wenxin Hu]
{
{\bf Wenxin Hu}\\
{\small University Bonn}\\[1ex]
}


\begin{frame}
    \titlepage
    \note{~}
\end{frame}


\AtBeginSection[]
{
  \begin{frame}
    \frametitle{Table of Contents}
    \tableofcontents[currentsection]
  \end{frame}
}

\section{Introduction}

\begin{frame}{Introduction}
    Expansion of FinTech Lending vs Traditional Banking
    \begin{figure}
        \centering
            \subfigure[Volumne]
             {\includegraphics[scale=0.35]{../../out/figures/figure2_lc_volume.eps}}
            \subfigure[Number]
             {\includegraphics[scale=0.35]{../../out/figures/figure2_lc_number.eps}}
        \caption{Lending Club loan volume and number development}
        \vspace{0.1cm}
        \hspace{0.25cm} \parbox{10cm}{\scriptsize The first panel illustrates the Lending Club loan volume growth (in billions of dollars) over the period 2007 to 2017. The second panel shows the respective quantity growth. Source: Lending Club.}
    \end{figure}
\end{frame}


\begin{frame}{Introduction}
    Identification Strategy
    \begin{itemize}
        \item Exploration of the geographic heterogeneity
        \item Panel data at "county" level, 2007-2017
        \item Fixed effect
    \end{itemize}
\end{frame}


\section{Hypotheses}

\begin{frame}{Hypotheses}
    Hypotheses in four aspects:
    \begin{enumerate}
        \item Financial crisis
        \item Market structure
        \item Technology
        \item Demographics
    \end{enumerate}
\end{frame}


\begin{frame}{Hypotheses}
    (1) Financial Crisis

    FinTech lending expanded faster in markets that were more affected by the Global Financial Crisis(+). 

    Measure: bank failure

    \begin{figure}
        \includegraphics[scale=0.5]{../../out/figures/figure5_failure.eps}
    \end{figure}
\end{frame}


\begin{frame}{Hypotheses}
    (2) Market structure (traditional banks)

    Fintech lending expanded faster in areas with low market concentration, which implies high market competition(-).

    Measure: Herfindahl-Hirschman-Index(HHI)

    \begin{figure}
        \includegraphics[scale=0.5]{../../out/figures/figure4_hhi.eps}
    \end{figure}
\end{frame}



\begin{frame}{Hypotheses}
    (3) Technology

    Fintech lending expanded faster where people have easier accesses to Internet. 

    Measure: Internet penetration
\end{frame}


\begin{frame}{Hypotheses}
    (4) Demographics

    The expansion of the fintech lending is faster in areas with more educated, young population, higher unemployment rate and lower personal income. 

    Measure: multiple social economic factors
\end{frame}


\section{Data}

\begin{frame}{Data}
    Dependent Variables
    \begin{tabular}{|p{4cm}|p{6cm}|}
        \hline
        \textbf{Variable} & \textbf{Definition and data source}  \\ 
        \hline
        Fintech Lending (volume) & Logarithm of loan volume from Lending Club at county level(dollars).   \\
        log(loan\_amnt) & Source: Lending Club   \\ 
        \hline
        Fintech Lending (number)  & Logarithm of the number of loans from Lending Club at county level.    \\
        log(loan\_no) & Source: Lending Club   \\
        \hline
        Traditional Banking Loan (Total) & Logarithm of total household debt balance in traditional banking (dollars).     \\ 
        log(bank\_loan) & Source: FRBNY Consumer Credit Panel / Equifax   \\ 
        \hline
        Traditional Banking Loan (Credit Card) & Logarithm of credit card debt balance in traditional banking (dollars).    \\ 
        log(bank\_loan) & Source: FRBNY Consumer Credit Panel / Equifax   \\ 
        \hline
    \end{tabular}
\end{frame}


\begin{frame}{Data}
    Independent Variables
    \centering
    \scalebox{0.7}{
    \begin{tabular}{|p{3cm}|p{7cm}|p{2cm}|}
        \hline
        \textbf{Variable} & \textbf{Definition and data source} & \textbf{Expected sign} \\ 
        \hline
        \multicolumn{3}{|l|}{\textbf{Financial crisis}} \\ \hline
        failure & Number of bank failures at county level. Source: Summary of Deposits.&  + \\ \hline
        \multicolumn{3}{|l|}{\textbf{Market Structure}} \\ \hline
        hhi & HHI for market concentration at county level. Source: Summary of Deposits.&  - \\ \hline
        \multicolumn{3}{|l|}{\textbf{Technology}} \\ \hline
        internet & Uses the Internet. Source: NTIA. &  + \\ \hline
        \multicolumn{3}{|l|}{\textbf{Socio-economic Demographics}} \\ \hline
        income & Per capita personal income(thousand of dollars). Source: DepaU.S. Bureau of Economic Analysis. &  ? \\ \hline
        unemployment & Unemployment rate(\%). Source: Bureau of Labor Statistics.&  + \\ \hline
        bachelor & Propotion of the population with bachelor degree(\%). Source: US Census Bureau.&  + \\ \hline
        young & Propotion of young population(age between 18 and 24)(\%). Source: US Census Bureau. &  + \\ \hline
        poverty & Proportion of population below poverty level. Source: US Census Bureau. & ? \\\hline
    \end{tabular}}
\end{frame}


\begin{frame}{Data}
    Summary Statistics
    \begin{center}
    \scalebox{0.8}{
    \begin{tabular}{l c c c c c}
        \input{../../out/tables/table1_sumstat.tex}
    \end{tabular}}
    \end{center}
\end{frame}


\section{Analysis}

\subsection{Drivers of FinTech Lending}

\begin{frame}{Drivers of FinTech Lending}
    Regression Model:
    \begin{small}
        \begin{equation}\begin{split}
log(loan\_amount)_{i,j,t} = & \beta_0 + \tau_{y} + \alpha_{j} + \beta_1 financial\ crisis_{i,j,t} \\ 
+ &  \beta_2 market\ structure_{i,j,t} + \beta_3 technology_{j,t}\\ 
+ &  \beta_4 demographics_{j,t} +\epsilon_{i,j,t},
     \end{split}\end{equation}

    \end{small}
\end{frame}


\begin{frame}{Regression Result}
    \centering
    \scalebox{0.6}{
    \begin{tabular}{l c c c c}
        \begin{equation}\begin{split}
log(loan\_amount)_{i,j,t} = & \beta_0 + \tau_{y} + \alpha_{j} + \beta_1 financial\ crisis_{i,j,t} \\ 
+ &  \beta_2 market\ structure_{i,j,t} + \beta_3 technology_{j,t}\\ 
+ &  \beta_4 demographics_{j,t} +\epsilon_{i,j,t},
     \end{split}\end{equation}

    \end{tabular}}
\end{frame}


\subsection{FinTech Lending vs Traditional Lending}

\begin{frame}{FinTech Lending vs Traditional Lending}
    Regression Model:
    \begin{tiny}
        \begin{equation}\begin{split}
log(loan\_amnt)_{i,j,t} - log(bank\_loan)_{j,t} = & \beta_0 + \tau_{y} + \alpha_{j} + \beta_1 financial\ crisis_{i,j,t} \\ 
+ &  \beta_2 market\ structure_{i,j,t} + \beta_3 technology_{j,t}\\ 
+ &  \beta_4 demographics_{j,t} +\epsilon_{i,j,t},
     \end{split}\end{equation}

    \end{tiny}
\end{frame}


\begin{frame}{Regression Result}
    \centering
    \scalebox{0.6}{
    \begin{tabular}{l c c c c c}
        \begin{equation}\begin{split}
log(loan\_amnt)_{i,j,t} - log(bank\_loan)_{j,t} = & \beta_0 + \tau_{y} + \alpha_{j} + \beta_1 financial\ crisis_{i,j,t} \\ 
+ &  \beta_2 market\ structure_{i,j,t} + \beta_3 technology_{j,t}\\ 
+ &  \beta_4 demographics_{j,t} +\epsilon_{i,j,t},
     \end{split}\end{equation}

    \end{tabular}}
\end{frame}


\subsection{Robustness Analysis}

\begin{frame}{Robustness Analysis I}
    Regression Model:
    \begin{scriptsize}
        \begin{equation}\begin{split}
log(loan\_amount)_{i,j,t} = & \beta_0 + \tau_{y} + \alpha_{j} + \beta_1 financial\ crisis_{i,j,t-1} \\ 
+ &  \beta_2 market\ structure_{i,j,t-1} + \beta_3 technology_{j,t-1}\\ 
+ &  \beta_4 demographics_{j,t-1} +\epsilon_{i,j,t-1},
     \end{split}\end{equation}

    \end{scriptsize}
\end{frame}


\begin{frame}{Regression Result}
    \centering
    \scalebox{0.35}{
    \begin{tabular}{l c c c c}
        \input{../../out/analysis/reg3_robustness.tex}
    \end{tabular}}
\end{frame}


\begin{frame}{Robustness Analysis II}
    Regression Model:
    \begin{footnotesize}
        \begin{equation}\begin{split}
log(loan\_amount)_{i,j,t}= & \beta_0 + \tau_{y} + \alpha_{j} + \beta_1 financial\ crisis_{i,j,t} \\ 
+ & \beta_2 market\ structure_{i,j,t} + \beta_3 technology_{j,t} \\ 
+ & \beta_4 demographics_{i,j,t}  \\
+ & \beta_5 financial\ crisis_{j,t} \times  market\ structure_{i,j,t} + \epsilon_{i,j,t},
     \end{split}\end{equation}

    \end{footnotesize}
\end{frame}


\begin{frame}{Regression Result}
    \centering
    \scalebox{0.35}{
    \begin{tabular}{l c c c c}
        \input{../../out/analysis/reg3_robustness.tex}
    \end{tabular}}
\end{frame}


\begin{frame}{Marginal Plots}
    \begin{figure}
        \centering
            \subfigure[Cond. marginal effect of bank failures on log(loan amount)]
             {\includegraphics[scale=0.35]{../../out/figures/figure1_margin_failure.eps}}
            \subfigure[Cond. marginal effect of HHI on log(loan amount)]
             {\includegraphics[scale=0.35]{../../out/figures/figure1_margin_hhi.eps}}
        \caption{Conditional marginal effects of Bank Failures and HHI}
        \vspace{0.1cm}
        \hspace{0.25cm} \parbox{10cm}{\scriptsize The estimated regression model is Equation 4. The dashed lines represent the mean minus standard
        deviation, mean, and mean plus standard deviation of the x-variable.}
    \end{figure}
\end{frame}


\section{Summary and Conclusion}

\begin{frame}{Summary and Conclusion}
    \begin{itemize}
        \item FinTech lending has been spurred by the financial crisis
        \item Entry of FinTech lending is constrained by the high market concentration
        \item States with higher share of young population and higher unemployment rate experience higher growth of FinTech lending
        \item Past technology may require some time to affect future FinTech lending positively
        \item The marginal effect of market structure on FinTech lending depends on financial crisis
    \end{itemize}
\end{frame}








\begin{frame}
    \frametitle{heading here}
    \begin{itemize}
        \item<+-> Please cite this template as: \citet{GaudeckerEconProjectTemplates}
        \item<+-> \citet{Schelling69} example in the code is taken from \citet{StachurskiSargent13}.
        \item<+-> The decision rule of an agent is \begin{align*}
    \text{move} & \quad \text{if} \quad n_\text{neighbours} < 4 \\
    \text{stay} & \quad \text{if} \quad n_\text{neighbours} \geq 4
\end{align*}

    \end{itemize}
    \note{~}
\end{frame}


% Print black screen only in presentation mode for finishing up.
\mode<beamer> {
    \beamersetaveragebackground{black}
    \begin{frame}
        \frametitle{}
    \end{frame}

    \beamersetaveragebackground{white}
}

\begin{frame}[allowframebreaks]
    \frametitle{References}
    
    
    \bibliography{refs}
    
\end{frame}

\end{document}
